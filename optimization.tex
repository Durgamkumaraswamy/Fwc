\documentclass{article}
\usepackage{enumitem}
\usepackage{amsmath}
\usepackage{tfrupee}
\usepackage{amsfonts}
\usepackage{float}
\usepackage{graphicx}
\graphicspath{{./images/}}
\begin{document}
\begin{enumerate} 
 \item A company prodces two types of goods,A and B,that require gold and silver.Each unit of type A requires $3$ g of silver and $1$ g of gold,while that of type B requires $1$ g of silver and $2$ g of gold.The company can use at the most $9$ g of silver and $8$ g of gold.If each unit of  type A brings a profit of \rupee~120 and that of type B \rupee~150, then find the number of units of each type that the company should produce to maximise profit.
		Formulate the above LPP and solve it graphically.Also,find the maximum profit.
\item Find the max/imum value of $7x+6y$ subject to the constraints :

		\begin{align}
			   x+y>=2 \\
			  2x+3y<=6 \\
			 x>=0{\text{ and }} y<=0 
		\end{align}
\item \begin{enumerate} %[label=(\alph*)]
			\item A window is in the form of a rectangle mounted by a semi-circular opening.The total perimeter of the window is $10$ m.Find the dimensions of the rectangular part of the window to admit maximum light through the whole opening.
			\item Divide the number $8$ into two positive numbers such that the sum of the cube of one and the square of the other is minimum.
	               \end{enumerate}
\item 
 Find the maximum and the minimum values of $$z=5x+2y$$ subject to the constraints:
	\begin{align}
		-2x-3y<=-6 \\
		   x-2y<=6 \\
		   6x+4y<=24 \\
		    -3x+2y<=3 \\
		    x>=0,y>=0 
	\end{align}
	\item Based on the given shaded region as the feasible region in the graph,at which point(s) is the objective function $$Z=3x+9y$$ maximum?
	\begin{figure}[H]
	\centering
		\includegraphics[width=\columnwidth]{figs/graph.jpg}
		\caption{graph}
		\label{fig:graph.jpg}
	\end{figure}
		\begin{enumerate}[label=(\Alph*)]
			\item Point B
			\item Point C
			\item Point D
			\item every point on the line segment CD
		\end{enumerate}
	\item The least value of the function {$f(x) = 2\cos x + x$} in the closed interval$[0,\frac{\pi}{2}]$
		\begin{enumerate}[label=(\Alph*)]
			\item $2$
			\item $\frac{\pi}{6}+\sqrt{3}$
			\item $\frac{\pi}{2}$
			\item The least value does not exist
		\end{enumerate}
	\item In the given graph,the feasible region for a LLP is shaded.The objective function 
	\begin{align}
		 Z = 2x-3y
	\end{align}
		,will be minimum at.
		\begin{figure}[H]
		\centering
			\includegraphics[width=\columnwidth]{figs/LPP.jpg}
			\caption{graph}
			\label{fig:LPP.jpg}
		\end{figure}
		\begin{enumerate}%[label=(\Alph*)]
			\item $$(4,10)$$
			\item $$(6,8)$$
			\item $$(0,8)$$
			\item $$(6,5)$$
		\end{enumerate}
	\item A linear programming problem is as follows:
		minimize 
	\begin{align}
		  Z = 30x+50y,
	\end{align}
		subject to the constraints,
	\begin{align}
		  3x+5y>=15\\
		 2x+3y<=18\\
		 x>=0,y>=0
	\end{align}
		In the feasible region the minimum value of Z occurs at
		\begin{enumerate}[label = (\Alph*)]
			\item a unique point
			\item no point
			\item infinitely many points
			\item two points only
		\end{enumerate}
	\item The area of a trapezium  defined by function f and given by 
		\begin{align}
		\text{$f(x) = (10+x)+\sqrt(100-x^2)$},
		\end{align}
		Then the area when it is maximised is:
		\begin{enumerate}[label=(\Alph*)]
			\item $75cm^2$
			\item $7\sqrt{3}cm^2$
			\item $75\sqrt{3}cm^2$
			\item $5cm^2$
		\end{enumerate}
	\item For an objective function \text{$Z =ax+by$}, where \text{$a,b >0$} the corner points of the feasible region is determined by set of constraints (linear inequalities are) $(0,20)$,$(10,10)$,$(30,30)$ and $(0,40)$ the condition a and b such that the maximum Z occurs at both the points $(30,30)$ and $(0,40)$ is;
		\begin{enumerate}[label=(\Alph*)]
			\item 
				$$-3a=0 $$
			\item 
				$$ a=3b $$
			\item 
				$$ a+2b=0 $$
			\item 
				$$ 2a-b=0 $$
		\end{enumerate}
	\item In a linear programming problem,the constraints on the decision variables x and y are 
		\begin{align}
		    x-3y>=0\\
		     y>=0,\\
		    0<=x<=3,
		\end{align}
		the feasible region 
		\begin{enumerate}[label=(\Alph*)]
			\item is not in the first quadrant
			\item is bounded in the first quadrant
			\item is unbounded in the first quadrant
			\item does not exist
		\end{enumerate}
 \end{enumerate}
 \end{document}
